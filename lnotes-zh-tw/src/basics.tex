\chapter{入門}

\section{Hello, World!}
\label{sec:hello_world}
把下面例子用編輯器保存為~\verb|hello_world.tex|,這就是一個最簡單的~\LaTeX~源文件。

\begin{code}
%hello_world.tex
\documentclass{article}
\begin{document}
    Hello, World!
\end{document}
\end{code}

有了源文件,我們可以在命令行把它編譯成~DVI~文件(DVI~格式見\ref{sec:dvi}小節)。此命令知道輸入的是~\LaTeX~源文件,所以這裡的~\verb|.tex|~後綴可以省略。以後的示例中可以省略的後綴都用~\verb|()|~標出,不再特別聲明。
\begin{code}
latex hello_world(.tex)
\end{code}

如果係統顯示類似下面的錯誤信息,請檢查源文件是否有拼寫錯誤。\verb|.log|~文件裡有更詳細的編譯信息。
\begin{code}
! LaTeX Error:
...
! Emergency stop.
...
No pages of output.
Transcript written on hello_world.log.
\end{code}

如果編譯成功,系統會報出類似下面的信息:
\begin{code}
Output written on hello_world.dvi (1 page, 232 bytes).
Transcript written on hello_world.log.
\end{code}

每種~\LaTeX~發行包附帶不同的~DVI~瀏覽器,比如~MiKTeX~的是~yap。
\begin{code}
yap hello_world(.dvi)
\end{code}

\section{格式及其轉換}
\subsection{頁面描述語言}
\label{sec:pdl}
頁面描述語言(Page Description Language,PDL)是一種在較高層次上描述實際輸出結果的語言。本文只討論其中三種與~\LaTeX~緊密相關的格式:DVI、PostScript、PDF。

\subsubsection{PostScript}
最早的打印機只用於打印字符,它使用的硬字符與打字機類似。後來出現的點陣(dot matrix)打字機用一系列的點來「畫」出字符,當然它也可以畫出圖形。當時向量圖的打印只能由繪圖儀(plotter)來完成。

1976~年,施樂(Xerox)推出了首台激光打印機,它結合了點陣打印機和繪圖儀的優點,可以同時打印高質量的圖形和文字。

同一時期,John Warnock~也在醞釀一種類似於~Forth~的圖形設計語言,也就是後來的~PostScript(PS),當時他正在舊金山一家電腦圖形公司~Evans \& Sutherland~工作。1978~年老闆想讓~Warnock~搬到位於猶他州的總部,他不想搬家就跳槽到了施樂。

Warnock~和~Martin Newell~開發了新的圖形系統~JaM(John and Martin),它後來被合併到施樂的打印機驅動程序~InterPress~中去。這兩位還開發過另一個系統~MaJ。

1982~年,Warnock~和施樂研究中心圖形實驗室主任~Chuck Geschke~一起離開施樂,成立了~Adobe~公司。Newell~後來也加入了~Adobe。

1984~年~Adobe~發佈~PS~後不久,Steve Jobs~跑來參觀,並建議用它來驅動激光打印機。次年,武裝著~PS~驅動的~Apple LaserWriter~橫空出世,打響了~80~年代中期桌面出版革命的第一槍。

90~年代中後期,廉價噴墨打印機的流行使得~PS~逐漸式微,因為~PS~驅動對它們畢竟是一個成本負擔。

\subsubsection{PDF}
1993~年,Adobe~推出了一種開放的格式:Portable Document Format(PDF),它於~2007~年成為~ISO 32000~標準。除了開放,PDF~比起~PS~還有一些其它優勢:
\begin{itemize}
    \item PDF~基本上是~PS~的一個子集,因此更輕便。
    \item PDF~可以嵌入更先進的字體,具體見\ref{sec:font}節。
    \item PDF~支持嵌入亂七八糟的東東,比如動畫。
    \item PDF~支持透明圖形。
\end{itemize}

PDF~雖然擁有上述優勢,起初它的推廣卻並不順利,因為其讀寫工具~Acrobat~太貴。Adobe~很快推出了免費的~Acrobat Reader(後更名為~Adobe Reader),並不斷改進~PDF,終於使它超越了曾經的事實標準~PS,成為網絡時代電子文檔的新標準。

\subsubsection{DVI}
\label{sec:dvi}
Knuth~最初設計的~\TeX~只能用於~XGP~打印機,這台打印機本身還需要一台~PDP-6~主機為它服務。1979~年,David Fuchs\footnote{Fuchs~本科畢業於普林斯頓,1978~年進入斯坦福攻讀博士學位。他不是~Knuth~的學生,但是完成過一些~\TeX~的開發任務。他在~Adobe~工作過一段時間,現在混入了娛樂圈,擔任過電影《Red Diaper Baby》和《Haiku Tunnel》的製片人。}提出把~\TeX~的輸出改為設備無關的格式,也就是~Device Independent format(DVI)。

DVI~只是一種中間格式,用戶還需要另外的處理程序(driver)把它轉換為其它格式,比如~PS~或~PDF~,甚至~PNG、SVG~等。DVI~不能嵌入字體和圖形,PS~和~PDF~可以選擇是否嵌入字體。

\subsubsection{Ghostscript}
\label{sec:ghostscript}
PS~輸出時需要一個解釋器(Raster Image Processor,RIP)來把它轉換為點陣圖形。RIP~可以是軟件,也可以是固件(firmware)或硬件\footnote{固件~RIP~在打印機內置處理器上運行,硬件~RIP~常見於高端打印設備。}。

Ghostscript~是一個基於~RIP~的軟件包,除了~RIP~它還有一些其它功能,比如處理~EPS,把~PS~轉換為~PDF~等。Ghostscript~已經被移植到~Windows、Unix/Linux、Mac OS~等多種作業系統,和它匹配的前端圖形用戶界面(GUI)有\href{http://pages.cs.wisc.edu/~ghost/}{GSview、Ghostview、gv}等。

\subsection{格式轉換}
\label{sec:convert_format}
DVI、PS、PDF~等格式的的轉換關係如\Fref{fig:convert_format}所示。

\begin{figure}[htbp]
\centering
\begin{tikzpicture}
    \node[box] (tex) {.tex};
    \node[box] (dvi) [right=5 of tex] {.dvi};
    \node[box] (pdf) [right=6 of dvi] {.pdf};
    \node[box] (ps) [above=3 of pdf] {.ps};
    \path (tex) edge [arrow] node[auto] {latex} (dvi)
        (dvi) edge [arrow] node[auto] {dvipdfm} (pdf)
        (dvi.north) [arrow,draw] to node[above,sloped] {dvips} (ps.west)
        (ps) edge [arrow] node[right] {ps2pdf} (pdf)
        (tex) edge [arrow,bloop] (pdf);
    \node [below=.7 of dvi] {pdflatex};
\end{tikzpicture}
\caption{格式轉換}
\label{fig:convert_format}
\end{figure}

最早的~driver~是~\verb|dvips|,它把~DVI~轉換為~PS。\verb|dvipdf|~把~DVI~轉為~PDF,它後來被~\verb|dvipdfm|~所取代;\verb|dvipdfm|~主要用於處理單字節字符,1999~年之後停止開發;在~\verb|dvipdfm|~基礎上發展來的~\verb|dvipdfmx|~可以處理多字節編碼(字符編碼詳見\ref{sec:encoding}節)。

pdf\TeX~是一種特殊的driver,它跳過~DVI,直接用~\TeX~源文件生成~PDF。基於~pdf\TeX~的~pdf\LaTeX~則把\LaTeX~源文件轉為~PDF。

包老師傾向於~\verb|dvipdfmx|,因為它對圖形格式的兼容性較好,而且擅長處理中文。

得到~DVI~後,我們可以在控制台用以下命令把它轉為~PDF。
\begin{code}
dvipdfm hello_world(.dvi)
\end{code}

我們也可以把它轉為~PS,接著用~Ghostscript~的一個命令行程序把它轉換為~PDF,注意第二個命令需要~\verb|.ps|~後綴。一般情況下不推薦這種方法,因為它多了個步驟。
\begin{code}
dvips hello_world(.dvi)
ps2pdf hello_world.ps
\end{code}

pdf\LaTeX~用法如下。
\begin{code}
pdflatex hello_world(.tex)
\end{code}

\section{\LaTeX~語句}
\LaTeX~源文件的每一行稱作一條語句(statement),語句可以分三種:命令(command)、數據(data)和註釋(comment)。

命令分為兩種:普通命令和環境(environment)。普通命令以\verb|\|~起始,大多隻有一行;而環境包含一對起始聲明和結尾聲明,用於多行的場合。命令和環境可以互相嵌套。

數據就是普通內容。註釋語句以~\verb|%|~起始,它在編譯過程中被忽略。

例如在\ref{sec:hello_world}節例1中,第一行是註釋,第二行是普通命令;第三、五行是環境的起始和結尾聲明;第四行是數據。

\section{文檔結構}
\subsection{文檔類、序言、正文}
\LaTeX~源文件的結構分三大部分,依次為:文檔類聲明、序言(可選)、正文。

文檔類聲明用來指定文檔的類型;序言(preamble)用來完成一些特殊任務,比如引入宏包,定義命令,設置環境等;文檔的實際內容則放在正文部分。這裡的正文指得是\verb|\begin{document}|和\verb|\end|~\verb|{document}|之間的部分,和通常人們心目中的「正文」概念有所出入。

這三部分的基本語法如下:
\begin{code}
\documentclass[options]{class}  %文檔類聲明
\usepackage[options]{package}   %引入宏包
...
\begin{document}                %正文
...
\end{document}
\end{code}

常用的文檔類(documentclass)有三種:\verb|article、report、book|,它們的常用選項見\fref{tab:class_options}。

\begin{table}[htbp]
\centering
\caption{文檔類常用選項}
\label{tab:class_options}
\begin{tabularx}{350pt}{lX}
    \toprule
    10pt, 11pt, 12pt & 正文字號,預設10pt。\LaTeX~會根據正文字號選擇標題、上下標等的字號。\\
    letterpaper, a4paper & 紙張尺寸,預設是~letter。\\
    notitlepage, titlepage & 標題後是否另起新頁。article~預設~notitlepage,report~和~book~預設有~titlepage。\\
    onecolumn, twocolumn & 欄數,預設單欄。\\
    oneside, twoside & 單面雙面。article~和~report~預設單面,book~預設雙面。\\
    landscape & 打印方向橫向,預設縱向。\\
    openany, openright & 此選項只用於~report~和~book。report~預設~openany~,book~預設~openright。\\
    draft & 草稿模式。有時某些行排得過滿,draft~模式可以在它們右邊標上粗黑線提醒用戶。\\
    \bottomrule
\end{tabularx}
\end{table}

\LaTeX~的核心只提供基本的功能,系統以宏包(package)的形式提供附加功能或增強原有功能。其它一些編程語言也有類似的模塊化機制,比如~C/C++~的~\verb|#include|,Java~的~\verb|import|。

\subsection{標題、摘要、章節}


一份文檔正文部分的開頭通常有標題、作者、摘要等信息,之後是章節等層次結構,內容則散佈於層次結構之間。

標題、作者、日期等命令如下,注意\verb|\maketitle|~命令要放在最後。
\begin{code}
\title{標題}
\author{作者}
\today
\maketitle
\end{code}

摘要環境用法如下:
\begin{code}
\begin{abstract}
...
\end{abstract}
\end{code}

常用的層次結構命令如下,
\begin{code}
\chapter{...}
\section{...}
\subsection{...}
\subsubsection{...}
\end{code}

每個高級層次可以包含若干低級層次。\verb|article|~中沒有~\verb|chapter|,而~\verb|report|~和~\verb|book|~則支持上面所有層次。

\subsection{目錄}

我們可以用~\verb|\tableofcontents|~命令來生成整個文檔的目錄,\LaTeX~會自動設定目錄包含的章節層次,也可以用~\verb|\setcounter|~命令來指定目錄層次深度。
\begin{code}
\tableofcontents
\setcounter{tocdepth}{2}
\end{code}

如果不想讓某個章節標題出現在目錄中,可以使用以下帶~\verb|*|~的命令來聲明章節。
\begin{code}
\chapter*{...}
\section*{...}
\subsection*{...}
\end{code}

類似地,我們也可以用以下命令生成插圖和表格的目錄,插圖和表格功能將在後面章節中介紹。

\begin{code}
\listoffigures
\listoftables
\end{code}

當章節或圖表等結構發生變化時,我們需要執行兩遍編譯命令以獲得正確結果。\LaTeX~之所以設計成這樣可能是因為當時的電腦內存容量有限。

\section{文字排版}
\subsection{字符輸入}
文檔中可以輸入的內容大致可以分為:普通字符、控制符、特殊符號、注音符號、預定義字符串等。而這些內容有兩種輸入模式:文本模式(預設)和數學模式,普通的行間(inline)數學模式用\verb|\$...\$|來表示。

\LaTeX~中有些字符(例如~\verb|# $ % ^ & _ { } ~ \|~等)被用作特殊的控制符,所以不能直接輸入,多數需要在前面加個~\verb|\|。而~\verb|\|~本身則要用~\verb|\textbackslash|~命令來輸入,因為~\verb|\\|~被用作了換行指令。很奇怪為什麼不用~C~語言的~\verb|\n|,也許是因為~\TeX~的編程語言是~Pascal。

\begin{code}
\# \$ \% \^{} \& \_ \{ \} \~{} \textbackslash
\end{code}

\Fref{tab:symbol}~提供了一些符號的輸入方法示例,完整的符號列表見~Scott Pakin的《The Comprehensive \LaTeX~ Symbol List》\citep{Pakin_2008}。

\begin{table}[htbp]
\centering
\caption{一些符號和預定義字符串}
\label{tab:symbol}
\begin{tabular}{llllll}
    \toprule
    \multicolumn{2}{c}{特殊符號} & \multicolumn{2}{c}{注音符號} & 
    \multicolumn{2}{c}{預定義字符串} \\
    \cmidrule(lr){1-2} \cmidrule(lr){3-4} \cmidrule(lr){5-6}
    \textcopyright  & \verb|\textcopyright|  & \aa & \verb|\aa| & 
        \today & \verb|\today| \\
    \textregistered & \verb|\textregistered| & \AA & \verb|\AA| & 
        \TeX & \verb|\TeX| \\
    $^\circ$C       & \verb|$^\circ$C|       & \ae & \verb|\ae| & 
        \LaTeX & \verb|\LaTeX| \\
    \textyen        & \verb|\textyen|        & \o  & \verb|\o| &
        \LaTeXe & \verb|\LaTeXe| \\
    \pounds         & \verb|\pounds|         & \"o & \verb|\"o| &
        \MF & \verb|\MF| \\
    \texteuro       & \verb|\texteuro|       & \^o & \verb|\^o| &
        \MP & \verb|\MP| \\
    \dots           & \verb|\dots|           & \~o & \verb|\~o| & \\
    \bottomrule
\end{tabular}
\end{table}

\subsection{換行、換頁、斷字}
通常~\LaTeX~會自動換行、換頁。用戶也可以用~\verb|\\|~或~\verb|\newline|~來強制換行;用~\verb|\newpage|~來強制換頁。

一般情況下~\LaTeX~會儘量均勻地斷字(Hyphenate),使得每一行的字間距分佈整齊。但有時我們也需要顯式指明斷字位置,比如下例就指明~BASIC~這個詞不能斷開,而~blar-blar-blar~可以在-處斷開。
\begin{code}
\hyphenation{BASIC blar-blar-blar}
\end{code}

\subsection{字樣、字號}

\LaTeX~會自動調整正文、標題、章節、上下標、腳註等的字樣\footnote{關於字樣詳見\ref{sec:typeface}節}、字號。我們也可以用\fref{tab:typeface_command}中的命令來設置字樣;用\fref{tab:fontsize_command}中的命令來設置相對字號,比如正文字號是~10pt、11pt、12pt~時,tiny的字號就分別是~5pt、6pt、6pt。

\LaTeX~有一個特別的字樣強調命令:\verb|\emph|,它在不同字樣和裝飾環境下有不同效果。比如周圍文字是正體,它就是斜體;反之它就是正體。

\begin{table}[hbtp]
\centering
\caption{字樣命令}
\label{tab:typeface_command}
\begin{tabular}{llll}
    \toprule
    \verb|\textrm{...}| & \textrm{roman} & 
    \verb|\textbf{...}| & \textbf{bold face} \\
    \verb|\textsf{...}| & \textsf{sans serif} & 
    \verb|\textit{...}| & \textit{italic} \\
    \verb|\texttt{...}| & \texttt{typewriter} & 
    \verb|\textsl{...}| & \textsl{slanted} \\
    \\
    \verb|\emph{...}|   & \emph{emphasized} & 
    \verb|\underline{...}|  & \underline{underline} \\
    \verb|\textsc{...}| & \textsc{Small Caps} & & \\
    \bottomrule
\end{tabular}
\end{table}

\begin{table}[htbp]
\centering
\caption{字號命令}
\label{tab:fontsize_command}
\begin{tabular}{llll}
    \toprule
    & \multicolumn{3}{c}{正文字號} \\
    \cmidrule(lr){2-4}
    命令 & 10pt & 11pt & 12pt \\
    \midrule
    \verb|\tiny|         & 5pt  & 6pt  & 6pt \\
    \verb|\scriptsize|   & 7pt  & 8pt  & 8pt \\
    \verb|\footnotesize| & 8pt  & 9pt  & 10pt \\
    \verb|\small|        & 9pt  & 10pt & 11pt \\
    \verb|\normalsize|   & 10pt & 11pt & 12pt \\
    \verb|\large|        & 12pt & 12pt & 14pt \\
    \verb|\Large|        & 14pt & 14pt & 17pt \\
    \verb|\LARGE|        & 17pt & 17pt & 20pt \\
    \verb|\huge|         & 20pt & 20pt & 25pt \\
    \verb|\Huge|         & 25pt & 25pt & 25pt \\
    \bottomrule
\end{tabular}
\end{table}

\section{常用命令環境}
\subsection{列表}

\LaTeX~中有三種列表環境:\verb|itemize、enumerate、description|,它們的一般用法如下:

\begin{demo}
\begin{itemize}
    \item C++
    \item Java
    \item HTML
\end{itemize}
\end{demo}

\begin{demo}
\begin{enumerate}
    \item C++
    \item Java
    \item HTML
\end{enumerate}
\end{demo}

\begin{demo}
\begin{description}
    \item{C++} 一種編程語言
    \item{Java} 另一種編程語言
    \item{HTML} 一種標記語言
\end{description}
\end{demo}

\subsection{對齊}
\LaTeX~中的段落預設兩端對齊(fully justified),我們也可以讓段落居左、居右或居中對齊。

\begin{demo}
\begin{flushleft}
本段落\\
居左
\end{flushleft}
\end{demo}

\begin{demo}
\begin{flushright}
本段落\\
居右
\end{flushright}
\end{demo}

\begin{demo}
\begin{center}
本段落\\
居中
\end{center}
\end{demo}

\subsection{摘錄}
\LaTeX~中有三種摘錄環境:\verb|quote、quotation、verse|。\verb|quote|~兩端都縮進,\verb|quotation|~在~\verb|quote|~的基礎上增加了首行縮進,\verb|verse|~比~\verb|quote|~多了第二行起的縮進。

\begin{demo}
正文
\begin{quote}
引文兩端都縮進。
\end{quote}
正文
\end{demo}

\begin{demo}
正文
\begin{quotation}
引文兩端縮進,首行縮進。
\end{quotation}
正文
\end{demo}

\begin{demo}
正文
\begin{verse}
引文兩端縮進,第二行起縮進。
\end{verse}
正文
\end{demo}

\subsection{原文照排}
一般文檔中,命令和源代碼通常使用等寬字樣來表示,也就是原文照排。對此~\LaTeX~提供了~\verb|\verb|~命令(一般用於在正文中插入較短的命令)和~\verb|verbatim|~環境。後者有帶~\verb|*|~的版本用來標明空格。

\begin{demo}
正文中插入\verb|command|
\begin{verbatim}
printf("Hello, world!");
\end{verbatim}
\begin{verbatim*}
printf("Hello, world!");
\end{verbatim*}
\end{demo}

\subsection{交叉引用}
我們常常需要引用文檔中~\verb|section、subsection、figure、table|~等對象的編號,這種功能叫作交叉引用(cross referencing)。

\LaTeX~中可以用~\verb|\label{marker}|~命令來定義一個標記,標記名可以是任意字符串,但是在全文中須保持唯一。之後可以用~\verb|\ref{marker}|~命令來引用標記處章節或圖表的編號,用~\verb|\pageref{marker}|~來引用標記處的頁碼。

\begin{demo}
被引用處\label{sec}\\
...\\
第\pageref{sec}頁\ref{sec}節
\end{demo}

文檔中新增交叉引用後,第一次執行~\verb|latex|~或~\verb|pdflatex|~編譯命令時會得到類似下面的警告信息。因為第一次編譯只會掃瞄出有交叉引用的地方,第二次編譯才能得到正確結果。

\begin{code}
LaTeX Warning: There were undefined references.
...
LaTeX Warning: Label(s) may have changed. Rerun to get cross-
references right.
\end{code}

\subsection{腳註}
腳註(footnote)的一般用法如下:
\begin{demo}
這裡是一段正文。\footnote{這裡是一段腳註。}
\end{demo}

\section{長度單位}
\LaTeX~中的常用長度單位如\Fref{tab:unit}~所示。point~是個傳統印刷業採用的單位,而~big point~是~Adobe~推出~PS~時新定義的單位。em~是個相對單位,比如當前字體是~11pt~時,1em~就是~11pt。
\begin{table}[htbp]
\caption{常用長度單位}
\label{tab:unit}
\centering
\begin{tabular}{llllll}
    \toprule
    in & 英吋 & pt & point, 1/72.27 in  & em & 當前字體中字母M的寬度 \\
    cm & 釐米 & bp & big point, 1/72 in & ex & 當前字體中字母X的高度 \\
    mm & 毫米 & pc & pica, 12 pt        & mu & math unit,1/18 em \\
    \bottomrule
\end{tabular}
\end{table}

\section{盒子}
\LaTeX~在排版時把每個對象(小到一個字母,大到一個段落)都視為一個矩形盒子(box),我們在~HTML~和~CSS~中也可以見到類似的模型。

\subsection{mbox~和~fbox}

\LaTeX~中最簡單的盒子是~\verb|\mbox|~和~\verb|\fbox|。前者把一組對象組合起來,後者在此基礎上加了個邊框。
\begin{demo}
\mbox{010 6278 5001}
\fbox{010 6278 5001}
\end{demo}

\subsection{makebox~和~framebox}
稍複雜的~\verb|\makebox|~和~\verb|\framebox|~提供了寬度和對齊方式控制選項。這裡用~l、r、s~分別代表居左、居右和分散對齊。
\begin{demo}
%語法:[寬度][對齊方式]{內容}
\makebox[100pt][l]{居左}
\framebox[100pt][r]{居右}
\end{demo}

\subsection{parbox~和~minipage}
大一些的對象比如整個段落可以用~\verb|\parbox|~命令和~\verb|\minipage|~環境,兩者語法類似,也提供了對齊方式和寬度的選項。但是這裡的對齊方式是指與周圍內容的縱向關係,用~t、c、b~分別代表居頂、居中和居底對齊。

\begin{demo}
%語法:[對齊方式]{寬度}{內容}
\parbox[c]{90pt}{錦瑟無端五十弦,\\一弦一柱思華年。}李商隱
\end{demo}

細心的讀者會發現~\verb|\parbox|~和~\verb|\minipage|~的選項排列順序和~\verb|\makebox|~和~\verb~\framebox|~的不一致,可能出自不同的作者。

\bibliographystyle{unsrtnat}
\bibliography{reading}
\newpage


