\chapter{表格}
\label{sec:tables}

\section{簡單表格}
\verb|tabular|~環境提供了最簡單的表格功能。它用~\verb|\hline|~命令代表橫線,\verb+|+~代表豎線,用~\verb|&|~來分欄。每個欄位的對齊方式可以用~l、c、r(左中右)來控制。
\begin{code}
\begin{tabular}{|l|c|r|}
    \hline 
    作業系統 & 發行版 & 編輯器 \\
    \hline 
    Windows & MikTeX & TeXnicCenter \\
    \hline 
    Unix/Linux & TeX Live & Emacs \\
    \hline 
    Mac OS & MacTeX & TeXShop \\
    \hline 
\end{tabular}
\end{code}

\begin{tabular}{|l|c|r|}
    \hline 
    作業系統 & 發行版 & 編輯器 \\
    \hline 
    Windows & MikTeX & TeXnicCenter \\
    \hline 
    Unix/Linux & TeX Live & Emacs \\
    \hline 
    Mac OS & MacTeX & TeXShop \\
    \hline 
\end{tabular}
\ \\

和針對插圖的~\verb|figure|~環境類似,\LaTeX~還有另一個針對錶格的浮動環境~\verb|table|。我們可以用它給上面的示例穿件馬甲,順便把表格簡化為科技文獻中常用的三線表。

\begin{code}
\begin{table}[htbp]
\caption{浮動環境中的三線表}
\label{tab:threesome}
\centering
\begin{tabular}{lll}
    \hline 
    作業系統 & 發行版 & 編輯器 \\
    \hline 
    Windows & MikTeX & TeXnicCenter \\
    Unix/Linux & TeX Live & Emacs \\
    Mac OS & MacTeX & TeXShop \\
    \hline 
\end{tabular}
\end{table}
\end{code}

\begin{table}[htbp]
\caption{浮動環境中的三線表}
\label{tab:threesome}
\centering
\begin{tabular}{lll}
    \hline 
    作業系統 & 發行版 & 編輯器 \\
    \hline 
    Windows & MikTeX & TeXnicCenter \\
    Unix/Linux & TeX Live & Emacs \\
    Mac OS & MacTeX & TeXShop \\
    \hline 
\end{tabular}
\end{table}

完美主義者可能覺得上面示例中的三條線一樣粗不夠美觀,這時可以使用~\verb|booktabs|~宏包\citep{Fear_2005}的幾個命令。

\begin{code}
\begin{table}[htbp]
\caption{浮動環境中的三線表}
\centering
\begin{tabular}{lll}
    \toprule
    作業系統 & 發行版 & 編輯器 \\
    \midrule
    Windows & MikTeX & TeXnicCenter \\
    Unix/Linux & TeX Live & Emacs \\
\end{code}

\begin{code}
    Mac OS & MacTeX & TeXShop \\
    \bottomrule
\end{tabular}
\end{table}
\end{code}

\begin{table}[htbp]
\caption{\texttt{booktabs}~宏包的效果}
\centering
\begin{tabular}{lll}
    \toprule
    作業系統 & 發行版 & 編輯器 \\
    \midrule
    Windows & MikTeX & TeXnicCenter \\
    Unix/Linux & TeX Live & Emacs \\
    Mac OS & MacTeX & TeXShop \\
    \bottomrule
\end{tabular}
\end{table}

\section{表格寬度}
有時我們需要控制某欄位寬度,可以將其對齊方式參數從~\verb|l、c、r|~改為~\verb|p{寬度}|~。
\begin{code}
\begin{table}[htbp]
\caption{控制欄位寬度}
\centering
\begin{tabular}{p{100pt}p{100pt}p{100pt}}
    \toprule
    作業系統 & 發行版 & 編輯器 \\
    \midrule
    Windows & MikTeX & TeXnicCenter \\
    Unix/Linux & TeX Live & Emacs \\
    Mac OS & MacTeX & TeXShop \\
    \bottomrule
\end{tabular}
\end{table}
\end{code}

\begin{table}[htbp]
\caption{控制欄位寬度}
\centering
\begin{tabular}{p{100pt}p{100pt}p{100pt}}
    \toprule
    作業系統 & 發行版 & 編輯器 \\
    \midrule
    Windows & MikTeX & TeXnicCenter \\
    Unix/Linux & TeX Live & Emacs \\
    Mac OS & MacTeX & TeXShop \\
    \bottomrule
\end{tabular}
\end{table}

若想控制整個表格的寬度可以使用~\verb|tabularx|~宏包,\verb|X|~參數表示某欄可以折行。

\begin{code}
\begin{table}[htbp]
\caption{控製表格寬度}
\centering
\begin{tabularx}{350pt}{lXlX}
    \toprule
    李白 & 平林漠漠煙如織,寒山一帶傷心碧。暝色入高樓,有人樓上愁。玉梯空佇立,宿鳥歸飛急。何處是歸程,長亭更短亭。& 
    泰戈爾 & 夏天的飛鳥,飛到我的窗前唱歌,又飛去了。秋天的黃葉,它們沒有什麼可唱,只嘆息一聲,飛落在那裡。\\
    \bottomrule
\end{tabularx}
\end{table}
\end{code}

\begin{table}[htbp]
\caption{控製表格寬度}
\centering
\begin{tabularx}{350pt}{lXlX}
    \toprule
    李白 & 平林漠漠煙如織,寒山一帶傷心碧。暝色入高樓,有人樓上愁。玉階空佇立,宿鳥歸飛急。何處是歸程,長亭更短亭。& 
    泰戈爾 & 夏天的飛鳥,飛到我的窗前唱歌,又飛去了。秋天的黃葉,它們沒有什麼可唱,只嘆息一聲,飛落在那裡。\\
    \bottomrule
\end{tabularx}
\end{table}

\section{跨行、跨列表格}
有時某欄需要橫跨幾列,我們可以使用~\verb|\multicolumn|~命令。它的前兩個參數指定橫跨列數和對齊方式。\verb|booktabs|~宏包的~\verb|\cmidrule|~命令用於橫跨幾列的橫線。
\begin{code}
\begin{table}[htbp]
\caption{跨欄表格}
\centering
\begin{tabular}{lll}
    \toprule
    & \multicolumn{2}{c}{常用工具} \\
    \cmidrule{2-3}
    作業系統 & 發行版 & 編輯器 \\
    \midrule
    Windows & MikTeX & TeXnicCenter \\
    Unix/Linux & TeX Live & Emacs \\
    Mac OS & MacTeX & TeXShop \\
    \bottomrule
\end{tabular}
\end{table}
\end{code}

\begin{table}[htbp]
\caption{跨欄表格}
\centering
\begin{tabular}{lll}
    \toprule
    & \multicolumn{2}{c}{常用工具} \\
    \cmidrule{2-3}
    作業系統 & 發行版 & 編輯器 \\
    \midrule
    Windows & MikTeX & TeXnicCenter \\
    Unix/Linux & TeX Live & Emacs \\
    Mac OS & MacTeX & TeXShop \\
    \bottomrule
\end{tabular}
\end{table}

跨行表格需要使用~\verb|multirow|~宏包,\verb|\multirow|~命令的前兩個參數是豎跨的行數和寬度。
\begin{code}
\usepackage{multirow}
...
\begin{table}[htbp]
\caption{跨行表格}
\centering
\begin{tabular}{lllc}
\end{code}
\begin{code}
    \toprule
    作業系統 & 發行版 & 編輯器 & 用戶體驗\\
    \midrule
    Windows & MikTeX & TeXnicCenter & 
    \multirow{3}{*}{\centering 爽} \\
    Unix/Linux & TeX Live & Emacs \\
    Mac OS & MacTeX & TeXShop \\
    \bottomrule
\end{tabular}
\end{table}
\end{code}

\begin{table}[htbp]
\caption{跨行表格}
\centering
\begin{tabular}{lllc}
    \toprule
    作業系統 & 發行版 & 編輯器 & 用戶體驗 \\
    \midrule
    Windows & MikTeX & TeXnicCenter & 
    \multirow{3}{*}{\centering 爽} \\
    Unix/Linux & TeX Live & Emacs \\
    Mac OS & MacTeX & TeXShop \\
    \bottomrule
\end{tabular}
\end{table}

\section{彩色表格}
彩色表格需要使用~\verb|colortbl|~宏包\citep{Carlisle_2001}提供的一些命令:\verb|\columncolor|、~\verb|\rowcolor|、\verb|\cellcolor|~等。
\begin{code}
\usepackage{colortbl}
...
\begin{table}[htbp]
\caption{彩色表格}
\centering
\begin{tabular}{lll}
    \toprule
    作業系統 & 發行版 & 編輯器 \\
    \midrule
    Windows & MikTeX & TeXnicCenter \\
\end{code}
\begin{code}
    \rowcolor[gray]{.8} Unix/Linux & TeX Live & Emacs \\
    Mac OS & MacTeX & TeXShop \\
    \bottomrule
\end{tabular}
\end{table}
\end{code}

\begin{table}[htbp]
\caption{彩色表格}
\centering
\begin{tabular}{lll}
    \toprule
    作業系統 & 發行版 & 編輯器 \\
    \midrule
    Windows & MikTeX & TeXnicCenter \\
    \rowcolor[gray]{.8} Unix/Linux & TeX Live & Emacs \\
    Mac OS & MacTeX & TeXShop \\
    \bottomrule
\end{tabular}
\end{table}

\section{長表格}
有時表格太長要跨頁,可以使用~\verb|longtable|~宏包\citep{Carlisle_2004}。\verb|\endfirsthead|、~\verb|\endhead|~命令用來定義首頁表頭和通用表頭,\verb|\endfoot|、\verb|\endlastfoot|~命令用來定義通用表尾和末頁表尾。
\begin{code}
\usepackage{longtable}
...
\begin{longtable}{ll}
\caption{長表格} \\
    \toprule
    作者 & 作品 \\
    \midrule
    \endfirsthead
    \midrule
    作者 & 作品 \\
    \midrule
    \endhead
    \midrule
    \multicolumn{2}{r}{接下頁\dots} \\
\end{code}
\begin{code}
    \endfoot
    \bottomrule
    \endlastfoot
    白居易 & 漢皇重色思傾國,\\
    & 御宇多年求不得。\\
    & 楊家有女初長成,\\ 
    & 養在深閨人未識。\\
    & 天生麗質難自棄,\\ 
    & 一朝選在君王側。\\
    & 回眸一笑百媚生,\\ 
    & 六宮粉黛無顏色。\\
    & 春寒賜浴華清池,\\ 
    & 溫泉水滑洗凝脂。\\
    & 侍兒扶起嬌無力,\\ 
    & 始是新承恩澤時。\\
    & 雲鬢花顏金步搖,\\ 
    & 芙蓉帳暖度春宵。\\
    & 春宵苦短日高起,\\ 
    & 從此君王不早朝。\\
\end{longtable}
\end{code}

\begin{longtable}{ll}
\caption{長表格} \\
    \toprule
    作者 & 作品 \\
    \midrule
    \endfirsthead
    \midrule
    作者 & 作品 \\
    \midrule
    \endhead
    \midrule
    \multicolumn{2}{r}{接下頁\dots} \\
    \endfoot
    \bottomrule
    \endlastfoot
    白居易 & 漢皇重色思傾國,\\
    & 御宇多年求不得。\\
    & 楊家有女初長成,\\
    & 養在深閨人未識。\\
    & 天生麗質難自棄,\\
    & 一朝選在君王側。\\
    & 回眸一笑百媚生,\\
    & 六宮粉黛無顏色。\\
    & 春寒賜浴華清池,\\
    & 溫泉水滑洗凝脂。\\
    & 侍兒扶起嬌無力,\\
    & 始是新承恩澤時。\\
    & 雲鬢花顏金步搖,\\
    & 芙蓉帳暖度春宵。\\
    & 春宵苦短日高起,\\
    & 從此君王不早朝。\\
\end{longtable}

\bibliographystyle{unsrtnat}
\bibliography{reading}
\newpage

