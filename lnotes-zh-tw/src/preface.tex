\chapter{序}

\begin{quotation}
滿紙荒唐言,一把辛酸淚!都雲作者痴,誰解其中味?\footnote{當年作博士論文時雖不曾增刪五次披閱十載,也被折磨得欲仙欲死,故與室友戲言將此五絕加入序言。多年以後的今天終於實現了此夙願。}
\begin{flushright}
--- 曹雪芹
\end{flushright}
\end{quotation}

最早聽說~\LaTeX~大約是~2002~年,一位同事演示了用它排版的一篇文章和幾幅圖。包老師\footnote{吾有多重人格,比如本色的是阿黃,下圍棋的是隱忍灰衣人,道貌岸然的是包老師。}不以為然,因為那些東西用~Microsoft Word~和~Visio~也可以做到,而且可以做得更快。再次聽說它是王垠同學在鬧退學,傳說他玩~Linux~和~\LaTeX~而走火入魔。

大約是~2005~年底,看了一下~lshort,用~\LaTeX~記了些數學筆記,開始有點感覺。包老師生性愚鈍,所以喜歡相對簡單的東西。HTML、Java~都用手寫,FrontPage、Dreamweaver、JBuilder~之類笨重的傢伙看兩眼就扔了,所以喜歡上~\LaTeX~只是時間問題。

次年老妻要寫博士論文,拿出~Word~底稿讓我排版。大家都知道~Word~太簡單了,誰都能用,但是不是誰都能用好。人稱電腦殺手的老妻製作的~Word~文檔自然使出了各種奇門遁甲,加上她實驗室、學校和家裡電腦裡的三個~EndNote~版本互不兼容,實在難以馴服。我只好重起爐灶,拿她的博士論文當小白鼠,試驗一下~\LaTeX~的威力。

就這樣接觸了兩三年,總算略窺門徑,感覺~\LaTeX~實在是博大精深,浩如煙海。而人到中年大腦儲存空間和處理能力都有點捉襟見肘,故時常作些筆記。一來對常用資料和問題進行彙編索引,便於查詢;二來也記錄一些心得。

日前老妻吵著要學~\LaTeX,便想這份筆記對初學者或有些許借鑑意義,於是系統地整理了一番,添油加醋包裝上市。

原本打算分九章,以紀念《九章算術》,實際上第八章完成時已如強弩之末,最後一章還須另擇黃道吉日。

本文第一章談談歷史背景;第二章介紹入門基礎;第三至五章講解數學、插圖、表格等對象的用法;第六章是一些特殊功能;第七、八章討論中文和字體的處理;第九章附加定製內容。

從難易程度上看前兩章較簡單,插圖、字體兩章較難。一般認為~\LaTeX~相對於微軟的傻瓜型軟件比較難學,所以這裡採取循序漸進,溫水煮青蛙的方法。

初則示弱,麻痹讀者;再則巧言令色,請君入甕;三則舌綻蓮花,誘敵深入;彼入得罄中則摧動機關,關門打狗;繼而嚴刑拷打,痛加折磨;待其意亂情迷徬徨無計之時,給予當頭棒喝醍醐灌頂,雖戛然而止亦餘音繞樑。

鄙人才疏學淺功力不逮,面對汗牛充棟罄竹難書\footnote{此處用法循阿扁古例。}的資料,未免考慮不周掛一漏萬,或有誤導,敬請海涵。若有高手高手高高手略撥閒暇指點一二,在下感激不盡\footnote{\href{mailto:huang.xingang@gmail.com}{huang.xingang@gmail.com}}
。

借此感謝一下老妻,如果不是伊天天看韓劇,包老師也不會有時間灌水和整理這份筆記。

